\chapter{Trabajo Relacionado}
Hacer una tesis implica encontrar una pregunta que valga la pena responder o un problema que valga la pena resolver y darle respuesta o solución. Es una tarea de investigación que tiene como aspecto muy importante conocer lo que ya existe alrededor de la pregunta o problema que se elige. 

Al llegar a este capitulo, el lector tiene una idea de cual es el problema. Seguramente se imagina problemas similares o soluciones al problema. El objetivo, en este momento, es convencerlo de que conocemos el problema y otros similares; que conocemos las formas en las que se lo ha intentado resolver (o a problemas similares); y que aún después de saber todo eso sigue siendo un problema importante, difícil y que nadie resolvió 8o nadie revolvió tan bien como nosotros).

Para escribir este capitulo hay que leer. Hay que buscar soluciones a problemas similares y compararlas con lo que nosotros queremos hacer. Si sabemos que la nuestra es mejor, ya podemos marcar cuales son los puntos débiles de las existentes. También se puede escribir un poco sobre otras investigaciones, que si bien no atacaron problemas parecidos, pueden ser aplicadas a resolver parte de este. 

Este capitulo es bueno ir escribiendolo en borrador cada vez que se lee algo (un articulo por ejemplo). Por lo menos hay que escribir un resumen de un párrafo de lo leído (registrando la referencia en el archivo bibliografia.bib y citando dede acá), y dar nuestra opinión al respecto en términos de su relación con el problema de nuestra tesis.

%Sección agregada por Diego Torres. Mayo 2019.
\section{Fuentes de Información}

Esta sección es una de las que demuestra más complejidad de escribir, al menos en la experiencia que tuve en la dirección de tesinas de grado en la Facultad de Informática de la UNLP, ya que la misma implica leer sobre trabajos relacionados y poder explicarlos dentro del contexto de vuestro trabajo. 

Leer implica también buscar dónde leer. A veces no es sencillo si no posees entrenamiento. Algo en lo que prestar mucha atención es en las fuentes de información. 

Particularmente prefiero las siguientes fuentes de información, porque considero que le dan mayor sustento a lo que justifiquemos e intentemos aportar al campo científico:
\begin{itemize}
    \item Artículos científicos de revistas (journals) o de conferencias. Ambos actualizados y en lo posible citados por otros trabajos. Esto último puede ser relativo al tiempo en que fueron publicados. 
    \item Libros. Referentes en la disciplina.
\end{itemize}

Por otro lado, trato de no incluir
\begin{itemize}
    \item Páginas web.
    \item Documentos sueltos que se encuentran por internet. Por ejemplo, artículos de divulgación.
\end{itemize}

Para poder encontrar artículos científicos, una herramienta muy útil es Google Scholar \footnote{\url{http://scholar.google.com}}. Es una versión del famoso buscador destinado a basar sus búsquedas en artículos científicos. 


\section{Citar y no copiar}

A medida que vamos leyendo sobre otros trabajos relacionados o sobre articulo que describen parte del contexto que le da forma al problema que resolvemos en esta tesis, nos vamos a dar cuenta que existen otras miradas y acercamientos a un mismo problema. Muchas veces, nos va a parecer que no existe otra forma mejor de explicar lo mismo si no es con las palabras que le dieron los autores de los artículos que leímos. 

Sin embargo, la tesina de grado es un espacio donde los autores (alumnos que van a recibirse de licenciados) deben producir el contenido de la misma. 

Las fuentes bibliográficas que leemos nos van a servir para diferentes cosas. Principalmente para darle sustento a algo que afirmamos, por ejemplo:

``sabemos que es posible algoritmicamente encontrar el camino de distancia mínima entre dos nodos pertenecientes a un grafo \cite{dijkstra1959note}''

La afirmación desde ``sabemos'' hasta ``grafo'' es una producción de quien escribió la frase, sin embargo esto esta fundado en el artículo escrito posiblemente por otra persona en la cita que se encuentra codificada luego, en este caso con \cite{dijkstra1959note} y que se listará al final en la sección de bibliografía.